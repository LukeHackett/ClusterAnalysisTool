\section{User Requirements}
\label{sec:userrequirements}
All objectives have been set by BlackBerry, and have been combined into a 
requirements list. In order to sure that the most important development of the 
final product is undertaken first, a prioritisation system will need to be 
used.

MoSCoW was first developed by Dai Clegg \citep{clegg94}, and divides 
requirements into four categories \citep{brennan09} as outlined below:
\begin{itemize}
  \item Must
  \item Should
  \item Could
  \item Won't
\end{itemize}

The o's within the acronym are added to allow for easy pronounceability, and 
have no other purpose other than this. Each category is described in further 
detail below, along with each of the requirements outlined by BlackBerry.


\subsection{Must}
The Must category describes a requirement that {\bfseries must be satisfied} in
the final product, for the project to be considered a success 
\citep{brennan09}. 

The must requirements for this product are outlined below:
\begin{itemize}
  \item Design and develop an algorithm to cluster GPS coordinates.
  \item Compare RAT footprints of two data sets (products or pins) with each 
        set being N weeks’ worth of data and highlight differences in the RAT 
        usage/drop/fails along the route.
  \item Compare MIX\_BAND (Frequency bands) footprints of two data sets and 
        highlight differences in usage/drops/fails between sets.
\end{itemize}


\subsection{Should}
The Should category describes a requirement 
that is classed as a {\bfseries high-priority item} that should be included in 
the final product if at all possible \citep{brennan09}. These requirements are
often critical, but can be satisfied in other ways if needed. 

The should requirements for this project are outlined below:
\begin{itemize}
  \item Compare two sets of call data and create drop/failure clusters, 
        highlighting the differences. 
  \item Ability to tag drops/fails with classification attributes and even 
        compare based on attributes.
\end{itemize}


\subsection{Could}
\label{sec:could}
The Could category describes a requirement that is considered {\bfseries 
desirable but not necessary} \citep{brennan09}. This will be included if 
resources permit (both time and technical/non-technical). 

The could requirements for this project are outlined below:
\begin{itemize}
  \item Velocity differences along route.
  \item Utilise a number of charting techniques to display the analysis 
        results.
  \item Plot a map for all call attempts and call ends (success/fail) for a 
        given period of time, for different devices.
  \item Allow the user to filter the clustering down (e.g. show only call 
        drops).
\end{itemize}


\subsection{Won't}
The Won't category describes a requirement that will not be implemented in a 
given release. This is usually in the form of an agreement between all the 
project stakeholders \citep{brennan09}. However the requirements may be 
implemented in the future. 

The won't requirements for this project are outlined below:
\begin{itemize}
  \item Integration with current internal BlackBerry testing systems.
\end{itemize}
