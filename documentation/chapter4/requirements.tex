\chapter{Requirements}
\label{chap:requirements}
There is a hierarchical approach that should be used in structuring 
requirements \citep{cadle10}. Table \ref{table:requirementsCategories} outlines
the main four categories:

\begin{table}[H]
  \begin{tabular}{|l|l|l|l|}
    \hline
    {\bf General} & {\bf Technical} & {\bf Functional} & {\bf Non-Functional} \\ 
    \hline
    Business constraints & Hardware & Data entry & Performance \\ 
    Business policies & Software & Data maintenance & Security \\ 
    Legal & Interoperability & Procedure & Legal and Access \\ 
    Branding & Internet & Retrieval & Backup and Recovery \\ 
    Cultural & ~ & ~ & Archiving and Retention \\ 
    Language & ~ & ~ & Maintainability \\ 
    ~ & ~ & ~ & Business Continuity \\ 
    ~ & ~ & ~ & Availability \\ 
    ~ & ~ & ~ & Usability \\ 
    ~ & ~ & ~ & Capacity \\
    \hline
  \end{tabular}
  \label{table:requirementsCategories}
\end{table}

Each of the four categories described in Table \ref{table:requirementsCategories}, 
will form the basis of this requirements chapter.


% General Requirements
\newpage
\section{General Requirements}
The general requirements are ones that define the business policies, standards 
and needs \citep{cadle10}. Many of the requirements are not related to this 
project, but will be noted for reference purposes.

\subsection{Business constraints}
The business constraints are the constraints that will directly effect this 
project. There are no constraints from BlackBerry, however there are time based 
constraints from the University.

\subsection{Business policies}
The business policies cover aspects such as business standards and business 
policy decisions. These are enforced to ensure consistency across is maintained 
across an organisation. This project has no requirements to deal with business 
policies.

\subsection{Legal}
\subsubsection*{Copyright, Designs and Patents}

In order to avoid various patent disputes, any third party resources, such as 
programming languages, development tools and external libraries will be only 
used if the licence associated with the item is of a non-proprietary agreement.

However there are some circumstances where by proprietary sources will have to 
be used for example the use of Visual Studio 2010. 

These sources will have been chosen to meet the interests of all parties 
involved, and where applicable the version of software will be taken into 
consideration. 

Any use of non-proprietary software must first be accepted by BlackBerry, to 
ensure that the correct licences are used.

All software released by this project will be under the New BSD Licence (also 
referred to as the BSD 3-Clause License). This license allows for 
re-distribution and use in both source and binary formats. 

The license also allows for the source code to be modified as long as the 
licence agreements are met.

\subsubsection*{Data Protection}
The Data Protection Act (1998) provides security for personal or business data 
that is stored electronically. In order for the project to continue, there will 
be various pieces of data required to fully test the system from BlackBerry. 
The data is to be transmitted via an encrypted form, and will not be passed on 
to any third parties. 

All data is of the property of BlackBerry unless otherwise stated.

\subsubsection*{Intellectual Property}
All intellectual property rights will belong to BlackBerry, once the project 
has been assessed by the university. This will allow for future modification 
and extension to the end deliverable from this project. Once the project is 
completed, the licence agreement can be updated to reflect BlackBerry’s 
business practice if required.

\subsection{Branding}
The branding requirement is concerned with the image and style promoted by 
BlackBerry. This would take into considerations such as style guides, fonts and 
logos. 

The end product is intended for internal use only (not for public use), and 
hence the branding requirements will not need to be followed.

\subsection{Cultural}
This section is devoted to BlackBerry's culture, such as management styles. As 
the project is not being developed within BlackBerry's facilities, the culture 
style is not required to be taken into consideration.

\subsection{Language}
This project will be completed utilising the English (British) language. 
Additional languages will not be supported.

% Technical Requirements
\newpage
\section{Technical Requirements}
Technical requirements state the technical policies and constraints to be 
adopted from BlackBerry \citep{cadle10}.

\subsection{Hardware}
The end product will not interact with any bespoke hardware, and therefore there 
are no hardware requirements.

\subsection{Software}
The end product should be able to run upon an existing internal server. The 
server in question, is a Microsoft Windows based operating system.

From talks with BlackBerry, it is apparent that there are a number of languages 
and tools that can be used to create the final product. However due to the 
Windows based environment that the end software product will be running on, it
was recommended that C\# would be used.

The advantage of using C\#, is the fact that C\# runs upon the Microsoft .NET 
framework. This means that as long as the production machine is running the 
same version of the .NET framework as the development machine, the tool should 
work with few issues. 

This is obviously a major advantage, as additional compilers, interpreters or 
parsers are simply not required. There would also be little setup overheads as
well.

\subsection{Interoperability}
The final software product is not expected to interact with BlackBerry's 
internal systems. The product will handle the importing of data from a given 
file. This is discussed in further detail in the Functional Requirements 
section. The product will not be required to communicate with other pieces of 
hardware.

\subsection{Internet}
The final software product will not require an internet connection in order to
work, and therefore BlackBerry's networking policies will not need to be taken
into consideration.

% Functional Requirements
\newpage
\section{Functional Requirements}
The functional requirements of a system are those that set out the features 
that any software solution should provide \citep{cadle10}. This is only 
intended to be a brief overview, a more detailed specification can be found 
within the specification section of this document.

\subsection{Data entry}
The gathering and recording of data will be handled by BlackBerry. The end 
product will not in any way need to gather data, however it will need to 
process a set of data found within a given file.

\subsection{Data maintenance}
As the gathering and recording of data is handled by BlackBerry, the requests 
for additional data or for the data to be changed, must be first authorised by 
BlackBerry. 

\subsection{Procedure}
New features, or analysis procedures that are outside of the project 
specification must be first authorised for use by BlackBerry.

\subsection{Retrieval}
The retrieval of data is concerned with the accuracy and integrity of the data.
In order to achieve high accuracy and integrity additional internal knowledge 
is required that is unable to be shared with third parties of BlackBerry. It is
therefore assumed that all data pushed from BlackBerry is correct.

% Non-Functional Requirements
\newpage
\section{Non-Functional Requirements}
The non-functional requirements deal with how well the solution will operate 
\citep{cadle10}. 

\subsection{Performance}
The performance of the system should be good, however it must be noted that the
system is intended to run as a service upon a server. The output data and 
analysis will be used by additional services which are not part of this 
project.

\subsection{Security}
All data handled by this project is strictly confidential, and must not be 
distributed in any form. This project will not handle database storage, as the
product is intended to interact with BlackBerry's secure network. 

All output data will be stored in files, that will be stored upon a secure 
revision control system, with access through a secure shell.

\subsection{Legal and Access}
Access permissions to data (such as being able to delete a particular piece of 
data from a live database), is not a requirement of this project.

\subsection{Backup and Recovery}
The project or it's data is not in anyway to be stored upon a portable data 
storage device, nor should it be stored upon publicly accessible cloud storage.

The project will make use of the git subversion system over a secure shell 
connection. This will ensure that:
\begin{enumerate}
  \item A copy of the project is stored upon a remote, secure server;
  \item Changes are able to be tracked;
  \item Issues and comments are able to be raised in a secure environment.
\end{enumerate}

The above measures will in effect be able to reduce against the loss, damage or 
theft of sensitive company information.

\subsection{Archiving and Retention}
The University will temporarily retain the project (software product and the 
final report) for assessment purposes only. Once the assessment of the project 
is completed the University will release the project and will no longer hold a 
copy of the project.

A copy of the project (excluding any input data) will be taken by the student 
for professional development purposes. 

\subsection{Maintainability}
All maintenance and related extensions to the project will be handled by 
BlackBerry as of Friday, 26th April 2013. This deadline has been set by the 
University as the final hand in date for the project.

\subsection{Capacity}
The end product should be able to handle data from multiple files, that could be 
representing multiple weeks worth of data. 

Each file can have any number of clusterable data, however files with limited
data may not produce results.

% User Requirements
\newpage
\section{User Requirements}
\label{sec:userrequirements}
All objectives have been set by BlackBerry, and have been combined into a 
requirements list. In order to sure that the most important development of the 
final product is undertaken first, a prioritisation system will need to be 
used.

MoSCoW was first developed by Dai Clegg \citep{clegg94}, and divides 
requirements into four categories \citep{brennan09} as outlined below:
\begin{itemize}
  \item Must
  \item Should
  \item Could
  \item Won't
\end{itemize}

The o's within the acronym are added to allow for easy pronounceability, and 
have no other purpose other than this. Each category is described in further 
detail below, along with each of the requirements outlined by BlackBerry.


\subsection{Must}
The Must category describes a requirement that {\bfseries must be satisfied} in
the final product, for the project to be considered a success 
\citep{brennan09}. 

The must requirements for this product are outlined below:
\begin{itemize}
  \item Design and develop an algorithm to cluster GPS coordinates.
  \item Compare RAT footprints of two data sets (products or pins) with each 
        set being N weeks’ worth of data and highlight differences in the RAT 
        usage/drop/fails along the route.
  \item Compare MIX\_BAND (Frequency bands) footprints of two data sets and 
        highlight differences in usage/drops/fails between sets.
\end{itemize}


\subsection{Should}
The Should category describes a requirement 
that is classed as a {\bfseries high-priority item} that should be included in 
the final product if at all possible \citep{brennan09}. These requirements are
often critical, but can be satisfied in other ways if needed. 

The should requirements for this project are outlined below:
\begin{itemize}
  \item Compare two sets of call data and create drop/failure clusters, 
        highlighting the differences. 
  \item Ability to tag drops/fails with classification attributes and even 
        compare based on attributes.
\end{itemize}


\subsection{Could}
\label{sec:could}
The Could category describes a requirement that is considered {\bfseries 
desirable but not necessary} \citep{brennan09}. This will be included if 
resources permit (both time and technical/non-technical). 

The could requirements for this project are outlined below:
\begin{itemize}
  \item Velocity differences along route.
  \item Utilise a number of charting techniques to display the analysis 
        results.
  \item Plot a map for all call attempts and call ends (success/fail) for a 
        given period of time, for different devices.
  \item Allow the user to filter the clustering down (e.g. show only call 
        drops).
\end{itemize}


\subsection{Won't}
The Won't category describes a requirement that will not be implemented in a 
given release. This is usually in the form of an agreement between all the 
project stakeholders \citep{brennan09}. However the requirements may be 
implemented in the future. 

The won't requirements for this project are outlined below:
\begin{itemize}
  \item Integration with current internal BlackBerry testing systems.
\end{itemize}


% Assumptions 
\newpage
\section{Assumptions}
The following assumptions will be made throughout this project:
\begin{itemize}
  \item The programmer has a limited knowledge of the mobile network;
  \item The input data generated by the client is correct.
\end{itemize}

