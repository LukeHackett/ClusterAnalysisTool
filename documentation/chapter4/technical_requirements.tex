\section{Technical Requirements}
Technical requirements state the technical policies and constraints to be 
adopted from BlackBerry \citep{cadle10}.

\subsection{Hardware}
The end product will not interact with any bespoke hardware, and therefore there 
are no hardware requirements.

\subsection{Software}
The end product should be able to run upon an existing internal server. The 
server in question, is a Microsoft Windows based operating system.

From talks with BlackBerry, it is apparent that there are a number of languages 
and tools that can be used to create the final product. However due to the 
Windows based environment that the end software product will be running on, it
was recommended that C\# would be used.

The advantage of using C\#, is the fact that C\# runs upon the Microsoft .NET 
framework. This means that as long as the production machine is running the 
same version of the .NET framework as the development machine, the tool should 
work with few issues. 

This is obviously a major advantage, as additional compilers, interpreters or 
parsers are simply not required. There would also be little setup overheads as
well.

\subsection{Interoperability}
The final software product is not expected to interact with BlackBerry's 
internal systems. The product will handle the importing of data from a given 
file. This is discussed in further detail in the Functional Requirements 
section. The product will not be required to communicate with other pieces of 
hardware.

\subsection{Internet}
The final software product will not require an internet connection in order to
work, and therefore BlackBerry's networking policies will not need to be taken
into consideration.