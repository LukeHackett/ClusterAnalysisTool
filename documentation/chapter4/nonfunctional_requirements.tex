\section{Non-Functional Requirements}
The non-functional requirements deal with how well the solution will operate 
\citep{cadle10}. 

\subsection{Performance}
The performance of the system should be good, however it must be noted that the
system is intended to run as a service upon a server. The output data and 
analysis will be used by additional services which are not part of this 
project.

\subsection{Security}
All data handled by this project is strictly confidential, and must not be 
distributed in any form. This project will not handle database storage, as the
product is intended to interact with BlackBerry's secure network. 

All output data will be stored in files, that will be stored upon a secure 
revision control system, with access through a secure shell.

\subsection{Legal and Access}
Access permissions to data (such as being able to delete a particular piece of 
data from a live database), is not a requirement of this project.

\subsection{Backup and Recovery}
The project or it's data is not in anyway to be stored upon a portable data 
storage device, nor should it be stored upon publicly accessible cloud storage.

The project will make use of the git subversion system over a secure shell 
connection. This will ensure that:
\begin{enumerate}
  \item A copy of the project is stored upon a remote, secure server;
  \item Changes are able to be tracked;
  \item Issues and comments are able to be raised in a secure environment.
\end{enumerate}

The above measures will in effect be able to reduce against the loss, damage or 
theft of sensitive company information.

\subsection{Archiving and Retention}
The University will temporarily retain the project (software product and the 
final report) for assessment purposes only. Once the assessment of the project 
is completed the University will release the project and will no longer hold a 
copy of the project.

A copy of the project (excluding any input data) will be taken by the student 
for professional development purposes. 

\subsection{Maintainability}
All maintenance and related extensions to the project will be handled by 
BlackBerry as of Friday, 26th April 2013. This deadline has been set by the 
University as the final hand in date for the project.

\subsection{Capacity}
The end product should be able to handle data from multiple files, that could be 
representing multiple weeks worth of data. 

Each file can have any number of clusterable data, however files with limited
data may not produce results.