\section{Primary Path Analysis}

In order to correctly and fully design a system as large as the one proposed, 
a sound understanding of the main process is required. The main process will 
be regarded as the scenario, which is ``a sequence of events or actions'' 
\citep{lunn03}. The scenario will contain various smaller scenarios which will 
be analysed separately. The main scenario is outlined below:

\begin{enumerate}
  \item Input data
  \item Cluster data
  \item Analyse the clustered data
  \item Output analysis
\end{enumerate}

Typically scenarios could be performed in any order, however the main scenario
outlined above can only be performed in one order. For example, in order for 
the analysis of the clustered data to happen, the data must first be clustered.

Another example is that data can only be clustered if data was given to the 
scenario in the first place, i.e. clustering of non-existent data could not 
happen. 

Now that the main scenario has been defined, the primary path of the entire 
system can now be formed. The primary path is a path through a system that is 
most commonly used, with a few variances \citep{lunn03}. 

The primary path for the Clustering Analysis Tool is outlined below:

\begin{enumerate}
  \item Import data set
  \item Cluster the data set
  \item Analyse clustered data set
  \item Create analysis charts
\end{enumerate}

The primary path outlined above is very generic with regards clustering of the 
the data set. The reason for this is that the clustering upon the data set can 
be applied in more than one way. 

Firstly the data can be clustered upon a week by week basis. This means that 
when analysed, the time stamp of the event is taken into consideration. This 
could result in a cluster that was found during multiple spanning weeks.

Finally, the data can be clustered upon a product basis. This means that when 
analysed, the device name (product name) is taken into consideration. This 
could result in multiple week's worth of testing being grouped by the device 
name rather than by the time stamp. 

Both of these ``in-depth'' paths can be thought of as alternative paths. The 
alternative path will contain subtle changes to the primary path 
\citep{lunn03}. The alternative paths for both a week by week analysis and a 
device based analysis can be seen shown below.

\subsection*{Week-by-Week Analysis}
\begin{enumerate}
  \item[2.1.1] Cluster the data set as an entire entity
  \item[2.1.2] Group the clustered data by the week numbers
\end{enumerate}

\subsection*{Product-based Analysis}
\begin{enumerate}
  \item[2.1.1] Cluster the data set as an entire entity
  \item[2.1.2] Group the clustered data by the device name
\end{enumerate}
~\\
The primary and alternative paths outlined above are a relatively simple, and 
yet powerful way of applying the general ideas behind the system. Both of 
these paths will be actively referred to in both the activity diagrams and use 
case diagrams found in the following sections.