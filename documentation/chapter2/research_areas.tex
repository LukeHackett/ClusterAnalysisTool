\section{Research Areas}

Based upon the previous problem environment definition and the problem 
definition a larger survey was able to be conducted.

\subsection{Initial Survey}
An initial survey was originally conducted which would try to determine the 
research paths that should be taken. The survey should also try to determine if
the problem has already been solved before.

Unfortunately due to the vast secrecy of the mobile computing industry it was
unclear whether or not this problem has already been solved before. 

This was mainly down to the fact that device manufacturers do not release key
performance indicators about their devices upon particular networks, as this 
information is particularly damaging to the device manufacturers and the mobile
networks operators.

However it is highly unlikely that the objects that have been set by client are
the same objectives as any previous or current system. 

Even though there is no evidence to suggest that there is (or isn't) a similar 
tool already available, the problem can sill be solved with standard computing 
practices.

\subsection{Clustering}
The problem definition (section \ref{sec:defining_the_problem}) described the 
solution to the outlined problems as ``grouping the call logs''. In order to 
gain valid, comparable groupings a standardised methodology should be used. 
This field of computing is known as cluster analysis, and is part of the larger 
data mining sub-field.

As each route is fixed it is possible to compare various weeks' testing to 
other weeks. However it must be noted that this project will only focus upon 
clustering the `Call Drop' and `Setup Failure' call logs. The reason for this 
is that successful calls do not provide much analysis, in comparison to a 
failed call.

\subsection{Visualisation}
The problem definition also mentions the fact that once the `grouping of data' 
is completed, some form of visual representation of the groupings is required. 
Data Visualisation is concerned with trying to obtain information from a set 
of data. 

As cluster analysis forms sets of `similar' data, it would certainly seem 
logical at this stage to ensure that both clustering and data visualisation 
techniques are covered.

\subsection{The Mobile Network}
The basics of the mobile network should also be researched. This will allow all
readers of this project to gain an understanding of some of the technical 
terminologies associated with this project.

In summary, the following areas will be researched into further to help gain a
more vivid understanding the problem at large.

\begin{itemize}
  \item Mobile Network
  \item Clustering Algorithms
  \item Visualisation of Data
\end{itemize}