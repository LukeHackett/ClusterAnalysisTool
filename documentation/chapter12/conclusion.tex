\chapter{Conclusion}
\label{cha:conclusion}

During this report it was shown that it is possible to cluster a set of call 
usage data to obtain some useful analysis. 

Although the clustering is only completed upon one set of attributes within the
dataset, the additional analysis manages to highlight similarities and 
differences between each cluster.

Clustering call logs is probably nothing new (although this is unconfirmed due
to the secrecy of the industry). However trying to highlight the similarities 
and differences in device performance figures is a common interest within the
industry.

The research highlighted that there are a number of common techniques that can 
be applied to a set of data to distinguish a set of clusters. Each of these 
techniques has various advantages and disadvantages over their counter 
techniques.

Once the analysis process has completed, the results are represented as 
numerous in-memory data structures (before being saved to disk). This allows 
for logical access to the specifics and generalisations of the analysis 
relatively simply.

Each class was tested utilising Unit Testing, with the final complete system 
being testing via a basic smoke test. Dynamic testing of the system occurred 
upon the fly as the tool was being developed, which enabled the overall testing
time to decrease.

This project has shown that a Cluster Analysis Tool can be successfully 
implemented utilising standard computing knowledge and principles. This is 
despite the fact that the data that the tool clustered and analysed could be 
regarded as `non-standard'.

Due to the nature of the project, it does obviously rely upon a fixed data 
structure, however it is possible to change and extend the data structure in 
the future should there be a requirement to do so.

Although this project has focused on the clustering and analysis of call data, 
the methodology used throughout this project has demonstrated that a large 
problem can be split up into smaller, more manageable tasks.
