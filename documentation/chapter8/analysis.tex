\section{Analysis}
The Analysis aspect of the project is split up across two libraries --- 
Analysis and Analysis.Heatmap.

The first library focuses upon providing analysis for clusters on a product 
level and a week level. The library also supports multi-product and multi-week 
analysis.

The second library focuses upon providing support for generating a heatmap from
a given set of events. The library is able to change the colour scheme of the 
generated heatmap if required to.


\subsection{Analysis}
The EventAnalysis class forms the basis of all types of analysis --- referred 
to as primary analysis. The primary analysis defines operations such as the 
number of events that happened within a given cluster.

The analysis also tries to group similar events together to further highlight 
any similarities (or differences) within an intra-cluster environment, and 
across multiple weeks.

The {\ttfamily GetRatCount()} and {\ttfamily GetMixBandCount()} methods will 
return the number of events that started upon the a given RAT or MixBand 
respectfully.

Only one of the methods have been shown below, to reduce the amount of 
repeatability, as both methods have a similar implementation.

~\\
{\bfseries EventAnalysis::GetRatCount()}
\lstset{style=pseudocode}
\begin{lstlisting}
  RETURN the count of all events that started on the given start RAT.
\end{lstlisting}

~\\
As an extension to the above methods, events can be grouped by their start and 
end values. The same methods -- {\ttfamily GetRatCount()} and 
{\ttfamily GetMixBandCount()} -- support two parameters one for the start value
and one for the end value. 

Only one of the methods have been shown below, to reduce the amount of 
repeatability, as both methods have a similar implementation.

~\\
{\bfseries EventAnalysis::GetMixBandCount()}
\lstset{style=pseudocode}
\begin{lstlisting}
  RETURN the count of all events that started on the given start 
         MixBand and ended upon the given end MixBand.
\end{lstlisting}

~\\
The previous methods returned the count of events based upon the parameters 
given. In order to retrieve the actual raw events the 
{\ttfamily GroupByStartRat()} and {\ttfamily GroupByStartMixBand()} can be 
used. 

These methods require the start RAT or start MixBand respectfully. The 
pseudocode implementation of the {\ttfamily GroupByStartRat()} method is shown 
below.

~\\
{\bfseries EventAnalysis::GroupByStartRat()}
\lstset{style=pseudocode}
\begin{lstlisting}
  RETURN a key/value pair list of all events where:
            the key is the start RAT 
            the value is a list of events with the key start RAT.
\end{lstlisting}

~\\
As with the previous methods both the {\ttfamily GroupByStartEndRat()} and 
{\ttfamily GroupByStartEndMixBand()} methods support two parameters to enable
a more in depth grouping.

The pseudocode implementation of the {\ttfamily GroupByStartEndMixBand()} 
method is shown below.

~\\
{\bfseries EventAnalysis::GroupByStartEndMixBand()}
\lstset{style=pseudocode}
\begin{lstlisting}
  RETURN a key/value pair list of all events where:
            the key is the concatenation of the start MixBand and 
            end MixBand
            the value is a list of events with the given start MixBand 
            and end MixBand.
\end{lstlisting}


\subsection{Heatmap}
The Heatmap class provides various methods and functionality to be able to 
create a heat map based upon an input set of events (and their coordinates).

The design of the class is a merge between gHeat \citep{gHeat} and a bespoke 
guide written by \citeauthor{vester} \citep{vester}. The main reason for 
merging the two concepts together, was to reduce the amount of code required to
create an accurate heatmap.

The main problem with the GHeat library is that it was originally designed for 
integration with a Graphical User Interface, rather than outputting to an image 
which was the objective in this project.

The guide written by \citeauthor{vester} manages to produce a simple heatmap 
with varying colour options, however it wasn't as accurate as the gHeat 
library. By merging these two libraries together, it formed the basis of a 
powerful, and relatively simple set of heatmap classes.

The main entry point to the Heatmap is through the {\ttfamily GenerateHeatMap()}
method. This method will initialise a new blank image, and will convert each 
given event (with a coordinate) to a heatmap.

~\\
{\bfseries Heatmap::GenerateHeatMap()}
\lstset{style=pseudocode}
\begin{lstlisting}
  // Initialise a new image
  CREATE blank image

  // Convert the long/lat values to image pixels
  CREATE an intensity map
 
  // Convert the grayscale intensity map to a coloured map
  CONVERT the intensity map into a colour heat map
\end{lstlisting}

~\\
In order to create a heatmap, the original heat point is plotted onto the 
image. This forms the centroid of the heatpoint. 

Next each point of the outer circumference is generated. This is based upon the
centroid location plus a prefixed radius value multiplied by the cosine of the 
current degree of the circle.

~\\
{\bfseries Heatmap::CreateIntensityMap()}
\lstset{style=pseudocode}
\begin{lstlisting}
  // Initialise a new image
  CREATE blank image and set background to white

  // Convert coordinate points to heat points
  CONVERT Coordinate points to HeatPoints

  // Loop over all HeatPoints
  FOR each heatpoint (HP) in HeatPoints
    
    // This is the radius of the heat point 
    SET radius equal to a given value

    // This will create a circle around HP of radius.
    FOR integer (I) 1..360
      // Determine the X location
      SET x as HP.X + radius * Cos(value of I in radians)

      // Determine the Y location
      SET y = HP.Y + radius * Sin(value of I in radians)

      // Write the new heatpoint to the image 
      CREATE a new heatpoint from x and y

    END FOR

  END FOR
\end{lstlisting}
