\section{KML File Handling}

As previously mentioned figure \ref{fig:NSKML} illustrates the namespace that 
handles the reading of KML files and writing of KML and KMZ files. Within this 
section the basics of these classes will be discussed.

\subsection{KML Reader}
The KML Reader class encapsulates an XmlTextReader object for which the class 
can be found within the {\ttfamily System.Xml} namespace. 

A KML file is a well formed XML file, and hence the reason why the 
{\ttfamily XMLTextReader} class was able to be used mostly for the parsing 
procedure.

The main method within the file is the {\ttfamily GetCallLogs()} method, which
will return an event collection of events from the given KML file. 

If an invalid KML file is found, or no placemarks are able to be fully 
retrieved from the KML file then the method will simply return an empty Event 
collection.

~\\
{\bfseries KMLReader::GetCallLogs()}
\lstset{style=pseudocode}
\begin{lstlisting}
  WHILE not at the end of the file
    // Obtain the Coordinate information
    READ the coordinate tag

    // Obtain the additional meta-data
    READ the device name tag
    READ the device pin tag
    READ the timestamp tag
    READ the reference tag
    READ the type tag
    READ the start rat tag
    READ the end rat tag
    READ the start mix band tag
    READ the end mix band tag
    READ the start rrc state tag

    // Create a new Event based upon the above information
    CREATE a new Event based upon the coordiante and meta-data

    // Add the parsed event
    ADD the event to the List of Events
  END WHILE

  RETURN List of Events
\end{lstlisting}


\subsection{KML Writer}
The KML Writer class encapsulates an {\ttfamily XmlTextWriter} object, for 
which the class can be found within the the {\ttfamily System.Xml} namespace. 

As previously mentioned, a KML file is a well formed XML file, and 
{\ttfamily XmlTextWriter} object can be used to generate a well formed KML 
file.

The main entry point for the KMLWriter class is the {\ttfamily GenerateKML()} 
method. This method is able to take a number of parameters in order to generate
a well formed KML file, and is able to perform multiple options based upon the 
given parameters.

~\\
{\bfseries KMLWriter::GenerateKML()}
\lstset{style=pseudocode}
\begin{lstlisting}
  // Create a new KML file at a given location
  CREATE new kml file

  // Insert the heatmap if given
  IF heatmap present
    ADD heatmap reference to the kml file
  END IF

  // Loop over each cluster
  FOR each cluster in clusters
    // Organise events into their clusters
    WRITE a new cluster folder

    // Loop over each event in the cluster
    FOR each event in cluster
      WRITE event to the cluster folder
    END FOR

  END FOR

  // Write noise data if available
  IF noise present
    // Organise noise events into a noise folder
    WRITE a new cluster folder

    // Loop over each noisy event
    FOR each event in noise
      WRITE event to the noise cluster folder
    END FOR

  END IF
\end{lstlisting}


\subsection{KMZ Writer}
A KMZ file is a compressed KML file. One the advantages of using a KMZ file 
over a KML file is that the file is able to store images embedded directly in
itself. This is unlike a KML file, where by an image must be referenced.

The KMZWriter class has a dependency upon the KMLWriter class. The main entry 
point to the KMZWriter class is through the {\ttfamily GenerateKMZ()} method.

The method will utilise the KMLWriter class to create the KML file, and will 
then place all generated files into the compressed KMZ file.

~\\
{\bfseries KMZWriter::GenerateKMZ()}
\lstset{style=pseudocode}
\begin{lstlisting}
  // Create a new temporary working directory
  CREATE new temporary directory

  // This will create a new KML file.
  // A heatmap may also be created
  CREATE kml file and SAVE into the temporary

  // Create the temporary archive
  COMPRESS the temporary directory into an archive

  // Forms the KMZ file
  RENAME the archive extension to .KMZ

  // Remove all working files and folders
  REMOVE the temporary directory
\end{lstlisting}
