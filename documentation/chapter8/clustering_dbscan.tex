\section{DBSCAN}

The DBSCAN class provides an interface to allow for clustering of a collection 
of events. The formal description of the DBSCAN methodology and algorithm can 
be found within section \ref{sec:DBSCAN}.

For reference purposes, the basic DBSCAN algorithm is outlined below:
\begin{enumerate}
  \item Label all points as core or noise points;
  \item Eliminate noise points;
  \item Put an edge between all core points that are within {\em Eps} of each 
        other;
  \item Make each group of connected core points into a separate cluster;
  \item Assign each border point to one of the clusters of its associated core 
        points.
\end{enumerate}


\subsection{Analyse Data}
The main entry point to start the algorithm is through the use of the 
{\ttfamily Analyse()} method. The algorithm will automatically determine if 
points are noise, and if noise points are found they are removed from the 
results.

There is no limit to the number of clusters that can be generated, however 
there is a minimum number of points that are required to form a cluster. This 
value is represented by the minPts value.

~\\
{\bfseries DBSCAN::Analyse()}
\lstset{style=pseudocode}
\begin{lstlisting}
  // The "current working" cluster
  cluster = null;

  // Loop over all unvisisted events
  FOR each unvisisted event (E) in the dataset (D)
    // Prevent a revisit of the current event
    SET E as visited

    // Obtain all neighbourhood events 
    NeighbourhoodEvents = all events within the EPS distance of E

    // Ensure that there are sufficent events
    IF the total amount of NeighbourhoodEvents < MinimumPoints
      SET E as NOISE
    else
      C = next cluster
      expand all the Neighbourhood events to determine similarity
    END IF

  END FOR
\end{lstlisting}
{\textsf \footnotesize File Source: src/Clusters/DBSCAN.cs }


\subsection{Neighbourhood Search}
The {\ttfamily GetRegion()} method will return all events that are within the 
eps-neighbourhood of a given event.

~\\
{\bfseries DBSCAN::GetRegion()}
\lstset{style=pseudocode}
\begin{lstlisting}
  // The "central", given event
  Evt = given event
  
  // Loop over all known events
  FOR each event (E) in dataset (D)

    // Calculate the distance between the events
    distance = distance from E to Evt

    // Add the event if it's deemed to be a neighbour
    IF eps >= distance
      ADD E to neighbours

  END FOR
\end{lstlisting}
{\textsf \footnotesize File Source: src/Clusters/DBSCAN.cs }


\subsection{Expanding a Cluster}
The {\ttfamily ExpandCluster()} method will expand each of the event 
neighbours, and all of their neighbours. 

Ultimately, this method will deduce which events are within the given event's 
EPS neighbourhood, and hence whether or not they belong to a new cluster.

~\\
{\bfseries DBSCAN::ExpandCluster()}
\lstset{style=pseudocode}
\begin{lstlisting}
  // The given event
  GivenEvt = given event

  // Add the current event to a new cluster
  ADD event to cluster (C)

  // Get the all of evt's neighbours.
  neighbours = get all events within GivenEvts region

  // Loop over each event in GivenEvt's neighbours
  FOR each event evt in neighbours

    // Visit the evt if it has not been visited
    if evt has not been visited

      // Mark as visited
      SET evt as visited
      
      // Obtain all the events within the new region
      evtNeighbours = get all events within evts region

      // Assume all are non-noise if greater than the MinimumPoints value
      IF the total amount of NeighbourhoodEvents >= MinimumPoints

        // Merge the two neighbours lists together
        ADD evtNeighbours onto the end of neighbours
      END IF

    END IF

    // If the event doesn't belong to a cluster add to a new cluster
    IF evt does not belong to a cluster

      // Add the event to the cluster
      add evt to C
    END IF

  END FOR
\end{lstlisting}
{\textsf \footnotesize File Source: src/Clusters/DBSCAN.cs }