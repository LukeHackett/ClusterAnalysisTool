\section{Unit Testing}
\label{sec:unit_testing}

In order to test the various back end processes and methods a number of Unit 
Tests have been developed.

The classes are designed to test the underlying system, and are not able to 
test how a user interacts with the system (though a shell). The testing of the 
User Interface will be discussed within section \ref{sec:walkthroughs}.

A listing of the automated unit test classes can be found within the supplied 
software CD-ROM.

\subsection{Unit Test Implementation}

All test classes extend a main {\ttfamily Test} class. The objective of this 
test class is to ensure various constants (such as root directories) are 
available to all tests.

The Test class does not serve any other purpose.

\subsection{Test Solution Summary}
The objective behind each of the tests was to firstly ensure that a various 
stages of method calls a given variable should be equal to another variable. 

Secondly the tests ensure that not only the output was correct, but in the case
of JSON file to ensure that the output was syntactically correct. In this 
instance, if a JSON file was not syntactically correct then it would cause the 
visualisation aspect of the product to fail.

It must be noted that the visualisation aspect of the product has not 
undertaken any form of visualisation. The reason being is that the 
visualisation aspect was an add-on to the main project. 

The visualisation web files (HTML, CSS and JavaScript) are effectively a 
framework that ``wraps around'' the JSON files --- hence why it is imperative 
that the JSON files are syntactically correct.

Table \ref{tab:test_class_summary} highlights the overall test results for each 
(unit test) class.

\begin{table}[h]
  \centering
  \begin{tabular}{|l|c|c|c|}
    \hline
    {\bfseries Test Class}   & {\bfseries Tests Passed} & 
    {\bfseries Tests Failed} & {\bfseries Tests Ignored} \\ 
    \hline
    CoordinateTest           & 5    & 0   & 0   \\ 
    CentroidTest             & 5    & 0   & 0   \\
    SphericalCoordinateTest  & 1    & 0   & 0   \\
    CoordinateCollectionTest & 5    & 0   & 0   \\
    EventCollectionTest      & 5    & 0   & 0   \\    
    KMLReaderTest            & 8    & 0   & 0   \\  
    KMLWriterTest            & 2    & 0   & 0   \\
    JSONWriterTest           & 6    & 0   & 0   \\
    DBSCANTest               & 1    & 0   & 0   \\
    DropAnalysisTest         & 6    & 0   & 0   \\ 
    FailAnalysisTest         & 6    & 0   & 0   \\ 
    WeekAnalysisTest         & 12   & 0   & 0   \\ 
    ProductAnalysisTest      & 12   & 0   & 0   \\ 
    MultiWeekAnalysisTest    & 2    & 0   & 0   \\
    MultiProductAnalysisTest & 2    & 0   & 0   \\
    \hline
  \end{tabular}
  \caption[Summary of Unit Test Results]
          {Summary of Unit Test Results}
  \label{tab:test_class_summary}
\end{table}


\subsubsection{Coordinate Test}

The {\ttfamily CoordinateTest} class tests the methods that are found 
within the Coordinate class. This class acts as a base class to other classes 
and hence will be tested multiple times.

The main objective of this unit test is to ensure that the conversion between 
decimal degrees to radians (and vice-versa) is performed correctly. It is these 
carnations that form the basis of many aspects of the tool.

Table \ref{tab:coordinate_test} highlights the Coordinate test results.

\begin{table}[h]
  \centering
  \begin{tabular}{|l|c|}
    \hline
    {\bfseries Test Method}     & {\bfseries Test Status} \\ 
    \hline
    TestToSphericalCoordinate   & {\bfseries \color{OliveGreen} PASS}   \\
    TestCoordinateInstantiation & {\bfseries \color{OliveGreen} PASS}   \\ 
    TestToString                & {\bfseries \color{OliveGreen} PASS}   \\
    TestLatitudeAsRadians       & {\bfseries \color{OliveGreen} PASS}   \\
    TestLongitudeAsRadians      & {\bfseries \color{OliveGreen} PASS}   \\
    \hline
  \end{tabular}
  \caption[Summary of the Coordinate unit test results]
          {Summary of the Coordinate unit test results}
  \label{tab:coordinate_test}
\end{table}


\subsubsection{Centroid Test}

The {\ttfamily CentroidTest} class tests the methods that are found 
within the Centroid class. This class extends the Coordinate class, and hence 
some testing will be a repeat of the {\ttfamily CoodinateTest}.

The main objective of this unit test is to ensure that the conversion between 
decimal degrees to radians (and vice-versa) is performed correctly. It will 
also ensure the radius value is being correctly used.

Table \ref{tab:centroid_test} highlights the Centroid test results.

\begin{table}[h]
  \centering
  \begin{tabular}{|l|c|}
    \hline
    {\bfseries Test Method}     & {\bfseries Test Status} \\ 
    \hline
    TestLatitudeAsRadians       & {\bfseries \color{OliveGreen} PASS} \\ 
    TestToSphericalCoordinate   & {\bfseries \color{OliveGreen} PASS} \\ 
    TestLongitudeAsRadians      & {\bfseries \color{OliveGreen} PASS} \\ 
    TestCoordinateInstantiation & {\bfseries \color{OliveGreen} PASS} \\ 
    TestToString                & {\bfseries \color{OliveGreen} PASS} \\
    \hline
  \end{tabular}
  \caption[Summary of the Centroid unit test results]
          {Summary of the Centroid unit test results}
  \label{tab:centroid_test}
\end{table}


\subsubsection{SphericalCoordinate Test}

The {\ttfamily SphericalCoordinateTest} class tests the methods that are 
found within the SphericalCoordinate class. This class extends the Coordinate 
class, and hence some testing will be a repeat of the {\ttfamily CoodinateTest}.

The main objective of this unit test is to ensure that a Spherical Coordinate 
can be formed correctly. This involves converting the various sub-components 
within a coordinate (longitude, latitude, elevation) into an x-axis, y-axis,
and a z-axis cartesian coordinate value.

Table \ref{tab:spherical_coordinate_test} highlights the SphericalCoordinate test results.

\begin{table}[h]
  \centering
  \begin{tabular}{|l|c|}
    \hline
    {\bfseries Test Method}  & {\bfseries Test Status} \\ 
    \hline
    CoordinateConversionTest & {\bfseries \color{OliveGreen} PASS} \\
    \hline
  \end{tabular}
  \caption[Summary of the SphericalCoordinate unit test results]
          {Summary of the SphericalCoordinate unit test results}
  \label{tab:spherical_coordinate_test}
\end{table}


\subsubsection{CoordinateCollection Test}

The {\ttfamily CoordinateCollectionTest} class tests the methods that are 
found within the CoordinateCollection class. This class extends the List 
class found within the {\ttfamily System.Collections. Generic} namespace.

The class largely overrides some standard features (such as adding and removing 
elements), however does feature an number of bespoke methods. The class will 
calculate the centroid of the set of coordinates every time a coordinate has 
been added or removed.

Additional non-standard methods such as splitting and merging collections will 
also be tested.

Table \ref{tab:coordinate_collection_test} highlights the CoordinateCollection 
test results.

\begin{table}[h]
  \centering
  \begin{tabular}{|l|c|}
    \hline
    {\bfseries Test Method} & {\bfseries Test Status} \\ 
    \hline
    AddCoordinateTest       & {\bfseries \color{OliveGreen} PASS}   \\ 
    SplitTest               & {\bfseries \color{OliveGreen} PASS}   \\ 
    AppendTest              & {\bfseries \color{OliveGreen} PASS}   \\ 
    UpdateCentroidTest      & {\bfseries \color{OliveGreen} PASS}   \\ 
    AddRangeTest            & {\bfseries \color{OliveGreen} PASS}   \\
    \hline
  \end{tabular}
  \caption[Summary of the CoordinateCollection unit test results]
          {Summary of the CoordinateCollection unit test results}
  \label{tab:coordinate_collection_test}
\end{table}


\subsubsection{EventCollection Test}

The {\ttfamily EventCollectionTest} class tests the methods that are found 
within the EventCollection class. This class extends the List class found
within the {\ttfamily System.Collections.Generic} namespace.

The class has a similar structure to the {\ttfamily CoordinateCollection} 
class, providing similar methods. However the main difference is the type of 
objects the hold.

Additional non-standard methods such as splitting and merging collections will 
also be tested.

Table \ref{tab:event_collection_test} highlights the EventCollection test 
results.

\begin{table}[h]
  \centering
  \begin{tabular}{|l|c|}
    \hline
    {\bfseries Test Method} & {\bfseries Test Status} \\ 
    \hline
    SplitTest               & {\bfseries \color{OliveGreen} PASS}   \\ 
    UpdateCentroidTest      & {\bfseries \color{OliveGreen} PASS}   \\ 
    AppendTest              & {\bfseries \color{OliveGreen} PASS}   \\ 
    AddEventTest            & {\bfseries \color{OliveGreen} PASS}   \\ 
    AddRangeTest            & {\bfseries \color{OliveGreen} PASS}   \\
    \hline
  \end{tabular}
  \caption[Summary of the EventCollection unit test results]
          {Summary of the EventCollection unit test results}
  \label{tab:event_collection_test}
\end{table}


\subsubsection{KMLReader Test}

The {\ttfamily KMLReaderTest} class tests the methods that are found within the
KMLReader class. This class parses a given KML file, and converts it to an
EventCollection.

In order to test the read method within this class, a number of KML files were 
given to the {\ttfamily KMLReader} class to ensure that the correct number of 
events were returned from each KML file.

Table \ref{tab:kml_reader_test} highlights the KMLReader test results.

\begin{table}[h]
  \centering
  \begin{tabular}{|l|c|}
    \hline
    {\bfseries Test Method} & {\bfseries Test Status} \\ 
    \hline
    ReadKMLFile\_L32        & {\bfseries \color{OliveGreen} PASS}   \\ 
    ReadKMLFile\_L33        & {\bfseries \color{OliveGreen} PASS}   \\ 
    ReadKMLFile\_L34        & {\bfseries \color{OliveGreen} PASS}   \\ 
    ReadKMLFile\_L35        & {\bfseries \color{OliveGreen} PASS}   \\ 
    ReadKMLFile\_T32        & {\bfseries \color{OliveGreen} PASS}   \\ 
    ReadKMLFile\_T33        & {\bfseries \color{OliveGreen} PASS}   \\ 
    ReadKMLFile\_T34        & {\bfseries \color{OliveGreen} PASS}   \\ 
    ReadKMLFile\_T35        & {\bfseries \color{OliveGreen} PASS}   \\
    \hline
  \end{tabular}
  \caption[Summary of the KMLReader unit test results]
          {Summary of the KMLReader unit test results}
  \label{tab:kml_reader_test}
\end{table}


\subsubsection{KMLWriter Test}

The {\ttfamily KMLWriterTest} class tests the methods that are found within the
KMLWriter class. This class will write a number of EventCollections and 
heatmaps to a given ouptut KML file.

The main method has been tested against the DBSCAN output which inducing using 
the noise event collection and omitting the noise event collection data.

Table \ref{tab:kml_writer_test} highlights the KMLWriter test results.

\begin{table}[h]
  \centering
  \begin{tabular}{|l|c|}
    \hline
    {\bfseries Test Method} & {\bfseries Test Status} \\ 
    \hline
    TestGenerateKMLNoNoise  & {\bfseries \color{OliveGreen} PASS}   \\ 
    TestGenerateKMLNoise    & {\bfseries \color{OliveGreen} PASS}   \\
    \hline
  \end{tabular}
  \caption[Summary of the KMLWriter unit test results]
          {Summary of the KMLWriter unit test results}
  \label{tab:kml_writer_test}
\end{table}


\subsubsection{JSONWriter Test}

The {\ttfamily JSONWriterTest} class tests the methods that are found within 
the JSONWriter class. This class will `convert' the various internal data 
structures to the JavaScript Object Notation (JSON) equivalent.

Many of the methods found within the {\ttfamily JSONWriterTest} class will call
each other a various stages, and hence multiple tests can occur of a given 
method.

Table \ref{tab:json_writer_test} highlights the JSONWriter test results.

\begin{table}[h]
  \centering
  \begin{tabular}{|l|c|}
    \hline
    {\bfseries Test Method}      & {\bfseries Test Status} \\ 
    \hline
    TestPrettifyJSON             & {\bfseries \color{OliveGreen} PASS} \\ 
    TestCreateJSONObjectElements & {\bfseries \color{OliveGreen} PASS} \\ 
    TestCreateJSONArrayList      & {\bfseries \color{OliveGreen} PASS} \\ 
    TestCreateJSONArrayParams    & {\bfseries \color{OliveGreen} PASS} \\ 
    TestCreateKeyValue           & {\bfseries \color{OliveGreen} PASS} \\ 
    TestCreateJSONObjectKeyValue & {\bfseries \color{OliveGreen} PASS} \\
    \hline
  \end{tabular}
  \caption[Summary of the JSONWriter unit test results]
          {Summary of the JSONWriter unit test results}
  \label{tab:json_writer_test}
\end{table}


\subsubsection{DBSCAN Test}

The {\ttfamily DBSCANTest} class tests the methods that are found within the 
DBSCAN class. This class will cluster a set of events, and because of this 
only the main entry point could be tested.

The test strategy will create a number dense clusters based upon random 
geographical coordinates. The test will succeed if the algorithm manages to 
`reproduce' these clusters --- i.e. the number of expected clusters is equal to
the number of actual clusters.

Table \ref{tab:dbscan_test} highlights the DBSCAN test results.

\begin{table}[h]
  \centering
  \begin{tabular}{|l|c|}
    \hline
    {\bfseries Test Method} & {\bfseries Test Status} \\ 
    \hline
    AnalyseTest             & {\bfseries \color{OliveGreen} PASS} \\
    \hline
  \end{tabular}
  \caption[Summary of the DBSCAN unit test results]
          {Summary of the DBSCAN unit test results}
  \label{tab:dbscan_test}
\end{table}


\subsubsection{DropAnalysis Test}
The {\ttfamily DropAnalysisTest} class tests the {\ttfamily DropAnalysis} 
class. As previously mentioned, the {\ttfamily DropAnalysis} class extends the
{\ttfamily Analysis}, and testing of this class will happen at the same time of
testing the {\ttfamily DropAnalysis} class.

Table \ref{tab:drop_analysis_test} highlights the DropAnalysis test results.

\begin{table}[h]
  \centering
  \begin{tabular}{|l|c|}
    \hline
    {\bfseries Test Method}    & {\bfseries Test Status} \\ 
    \hline
    GroupByStartRatTest        & {\bfseries \color{OliveGreen} PASS}  \\ 
    GroupByStartEndMixBandTest & {\bfseries \color{OliveGreen} PASS}  \\ 
    GroupByStartEndRatTest     & {\bfseries \color{OliveGreen} PASS}  \\ 
    GroupByStartMixBandTest    & {\bfseries \color{OliveGreen} PASS}  \\ 
    GroupByStartRRCStateTest   & {\bfseries \color{OliveGreen} PASS}  \\ 
    TestAnalysisLength         & {\bfseries \color{OliveGreen} PASS}  \\
    \hline
  \end{tabular}
  \caption[Summary of the DropAnalysis unit test results]
          {Summary of the DropAnalysis unit test results}
  \label{tab:drop_analysis_test}
\end{table}


\subsubsection{FailAnalysis Test}
The {\ttfamily FailAnalysisTest} class tests the {\ttfamily FailAnalysis} 
class. As well as the {\ttfamily DropAnalysis} class the {\ttfamily FailAnalysis} 
class extends the {\ttfamily Analysis}, and testing of this class will happen 
at the same time of testing the {\ttfamily FailAnalysis} class.

Table \ref{tab:fail_analysis_test} highlights the FailAnalysis test results.

\begin{table}[h]
  \centering
  \begin{tabular}{|l|c|}
    \hline
    {\bfseries Test Method}    & {\bfseries Test Status} \\ 
    \hline
    GroupByStartRRCStateTest   & {\bfseries \color{OliveGreen} PASS}  \\ 
    GroupByStartEndMixBandTest & {\bfseries \color{OliveGreen} PASS}  \\ 
    GroupByStartMixBandTest    & {\bfseries \color{OliveGreen} PASS}  \\ 
    TestAnalysisLength         & {\bfseries \color{OliveGreen} PASS}  \\ 
    GroupByStartRatTest        & {\bfseries \color{OliveGreen} PASS}  \\ 
    GroupByStartEndRatTest     & {\bfseries \color{OliveGreen} PASS}  \\
    \hline
  \end{tabular}
  \caption[Summary of the FailAnalysis unit test results]
          {Summary of the FailAnalysis unit test results}
  \label{tab:fail_analysis_test}
\end{table}


\subsubsection{WeekAnalysis Test}

The {\ttfamily WeekAnalysisTest} class tests the methods that are found within 
the WeekAnalysis class. This class will analyse the data based upon a week 
basis.

The majority of these methods return well formed data structures, and hence the 
test strategy will look an ensuring that the correct results are returned as 
well as ensuring the data structures are correct.

Table \ref{tab:week_analysis_test} highlights the WeekAnalysis test results.

\begin{table}[h]
  \centering
  \begin{tabular}{|l|c|}
    \hline
    {\bfseries Test Method}           & {\bfseries Test Status} \\ 
    \hline
    GetFailStartEndRatFiguresTest     & {\bfseries \color{OliveGreen} PASS} \\ 
    GetDropStartRatFiguresTest        & {\bfseries \color{OliveGreen} PASS} \\ 
    GetDropStartEndRatFiguresTest     & {\bfseries \color{OliveGreen} PASS} \\ 
    GetFailStartEndMixBandFiguresTest & {\bfseries \color{OliveGreen} PASS} \\ 
    GetDropStartMixBandFiguresTest    & {\bfseries \color{OliveGreen} PASS} \\ 
    GetDropStartEndMixBandFiguresTest & {\bfseries \color{OliveGreen} PASS} \\ 
    GetFailFiguresTest                & {\bfseries \color{OliveGreen} PASS} \\ 
    GetFailStartRRCStateFiguresTest   & {\bfseries \color{OliveGreen} PASS} \\ 
    GetFailStartRatFiguresTest        & {\bfseries \color{OliveGreen} PASS} \\ 
    GetFailStartMixBandFiguresTest    & {\bfseries \color{OliveGreen} PASS} \\ 
    GetDropFiguresTest                & {\bfseries \color{OliveGreen} PASS} \\ 
    GetDropStartRRCStateFiguresTest   & {\bfseries \color{OliveGreen} PASS} \\
    \hline
  \end{tabular}
  \caption[Summary of the WeekAnalysis unit test results]
          {Summary of the WeekAnalysis unit test results}
  \label{tab:week_analysis_test}
\end{table}


\subsubsection{ProductAnalysis Test}

The {\ttfamily ProductAnalysisTest} class tests the methods that are found 
within the ProductAnalysis class. This class will analyse the data based upon a 
product basis.

As with {\ttfamily ProductAnalysis} class, the {\ttfamily ProductAnalysis} 
class return well formed data structures. The objective of this test class is 
to ensure that the correct results are returned as well as ensuring the data 
structures are correct.

Table \ref{tab:product_analysis_test} highlights the ProductAnalysis test 
results.

\begin{table}[h]
  \centering
  \begin{tabular}{|l|c|}
    \hline
    {\bfseries Test Method}           & {\bfseries Test Status} \\ 
    \hline
    GetFailStartRRCStateFiguresTest   & {\bfseries \color{OliveGreen} PASS} \\ 
    GetDropStartMixBandFiguresTest    & {\bfseries \color{OliveGreen} PASS} \\ 
    GetFailStartEndRatFiguresTest     & {\bfseries \color{OliveGreen} PASS} \\ 
    GetDropStartRatFiguresTest        & {\bfseries \color{OliveGreen} PASS} \\ 
    GetDropFiguresTest                & {\bfseries \color{OliveGreen} PASS} \\ 
    GetDropStartEndMixBandFiguresTest & {\bfseries \color{OliveGreen} PASS} \\ 
    GetFailFiguresTest                & {\bfseries \color{OliveGreen} PASS} \\ 
    GetDropStartRRCStateFiguresTest   & {\bfseries \color{OliveGreen} PASS} \\ 
    GetFailStartRatFiguresTest        & {\bfseries \color{OliveGreen} PASS} \\ 
    GetFailStartMixBandFiguresTest    & {\bfseries \color{OliveGreen} PASS} \\ 
    GetFailStartEndMixBandFiguresTest & {\bfseries \color{OliveGreen} PASS} \\ 
    GetDropStartEndRatFiguresTest     & {\bfseries \color{OliveGreen} PASS} \\
    \hline
  \end{tabular}
  \caption[Summary of the ProductAnalysis unit test results]
          {Summary of the ProductAnalysis unit test results}
  \label{tab:product_analysis_test}
\end{table}



\subsubsection{MultiWeekAnalysis Test}

The {\ttfamily MultiWeekAnalysisTest} class tests the methods that are found 
within the MultiWeekAnalysis class. This class will analyse the data based upon
a week-by-week basis.

The {\ttfamily MultiWeekAnalysis} class is an extension to the 
{\ttfamily WeekAnalysis} class. The class is able to support multiple weeks
(rather than just one). 

Many of the methods are similar to the WeekAnalysis, and to reduce 
repeatability these methods will not be tested directly. However these methods 
will be tested indirectly when formulating the results.

Table \ref{tab:multiweek_analysis_test} highlights the MultiWeekAnalysis test 
results.

\begin{table}[h]
  \centering
  \begin{tabular}{|l|c|}
    \hline
    {\bfseries Test Method} & {\bfseries Test Status} \\ 
    \hline
    AnalyseWeekTest         & {\bfseries \color{OliveGreen} PASS} \\ 
    AnalyseWeeksTest        & {\bfseries \color{OliveGreen} PASS} \\
    \hline
  \end{tabular}
  \caption[Summary of the MultiWeekAnalysis unit test results]
          {Summary of the MultiWeekAnalysis unit test results}
  \label{tab:multiweek_analysis_test}
\end{table}


\subsubsection{MultiProductAnalysis Test}

The {\ttfamily MultiProductAnalysisTest} class tests the methods that are found 
within the MultiProductAnalysis class. This class will analyse the data based 
upon a product basis.

The {\ttfamily MultiProductAnalysis} class is an extension to the 
{\ttfamily ProductAnalysis} class. The class is able to support multiple 
products across multiple weeks. 

Many of the methods are similar to the ProductAnalysis, and to reduce 
repeatability these methods will not be tested directly. However these methods 
will be tested indirectly when formulating the results.

Table \ref{tab:multiproduct_analysis_test} highlights the MultiProductAnalysis 
test results.

\begin{table}[h]
  \centering
  \begin{tabular}{|l|c|}
    \hline
    {\bfseries Test Method} & {\bfseries Test Status} \\ 
    \hline
    AnalyseWeekTest         & {\bfseries \color{OliveGreen} PASS} \\ 
    AnalyseWeeksTest        & {\bfseries \color{OliveGreen} PASS} \\
    \hline
  \end{tabular}
  \caption[Summary of the MultiProductAnalysis unit test results]
          {Summary of the MultiProductAnalysis unit test results}
  \label{tab:multiproduct_analysis_test}
\end{table}
