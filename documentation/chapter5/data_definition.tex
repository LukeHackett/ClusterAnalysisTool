\section{Data Definition}

Within section~\ref{sec:userrequirements} (page~\pageref{sec:userrequirements}), 
the various system requirements were highlighted. However the data requirements 
were not discussed. 

In order to cluster the data, the tool will need to access the associated call 
logs. Due to legal reasons and a lengthy authorisation processes, connecting 
directly to a BlackBerry data source was deemed to be out of the scope of this
project. 

However in order to ensure the tool is correctly working, BlackBerry have 
agreed to supply some real, anonymous data for testing purposes. The data will 
be supplied within a set KML format.

The KML file format is ``used to display geographic data in an Earth browser 
such as Google Earth, Google Maps'' \citep{google:KML}. The KML structure is 
tag-based which contains various elements and attributes that can be nested, 
and is an XML extension.

{\bf All} tags are case-sensitive and {\bf must} appear exactly as they are 
listed in the KML API reference. Within a given element, tags must also appear 
in the same order that is shown in the API reference. 

An example snippet showing an Event is shown below.

% Define XML Colour Scheme
\definecolor{gray}{rgb}{0.4,0.4,0.4}
\definecolor{pink}{HTML}{008080}
\definecolor{darkblue}{rgb}{0.0,0.0,0.6}
\lstdefinelanguage{XML}
{
  basicstyle=\ttfamily\small,
  columns=fullflexible,
  showstringspaces=false,
  commentstyle=\color{gray}\upshape
  morestring=[s]{>}{<},
  morecomment=[s]{<?}{?>},
  stringstyle=\color{black},
  identifierstyle=\color{pink},
  keywordstyle=\color{darkblue},
  morekeywords={Placemark,Point,coordinates,ExtendedData,SimpleData,
                Document,name,kml}
}

% Import the KML Template
\begin{lstlisting}[language=XML]
<Placemark>
  <Point>
    <coordinates>-0.2341408,51.3653237,0</coordinates>
  </Point>
  <ExtendedData>
    <SimpleData name="device">9800</SimpleData>
    <SimpleData name="pin">03dcde00</SimpleData>
    <SimpleData name="timestamp">07/08/2012 00:00:00<SimpleData>
    <SimpleData name="reference">true</SimpleData>          
    <SimpleData name="type">drop</SimpleData>
    <SimpleData name="start_rat">WCDMA</SimpleData>
    <SimpleData name="end_rat">WCDMA</SimpleData>
    <SimpleData name="start_mix_band">10812</SimpleData>
    <SimpleData name="end_mix_band">10787</SimpleData>
    <SimpleData name="start_rrc_state">IDLE<SimpleData>
  </ExtendedData>
</Placemark>
\end{lstlisting}


Each Placemark represents an event, with each Placemark containing data about 
the location of the event, and additional ``meta-data''. The location of the 
data is stored within the Point tag, and the additional ``meta-data'' is stored 
within the ExtendedData tag. 

The coordinates are expressed as decimals, and follow the following template:

\begin{lstlisting}[language=XML]
  <coordinates>longitude,Latitude,Altitude</coordinates>
\end{lstlisting}

{\em Note: Data that sits in between the coordinates tag cannot have {\bf any} 
additional whitespace.}

The additional ``meta-data'' is stored as a list of SimpleData items. The 
order of the elements is imperative, and {\bf must} be maintained. Each of the 
SimpleData tags must be present, even if no data is stored. If no data is 
stored, then an empty tag must be present.

Each of the SimpleData tags are discussed in more detail below:
\begin{itemize}
  \item {\bf device} \\
        The Name of the device.
  \item {\bf pin} \\
        The Personal Identification Number (PIN) of the device.
  \item {\bf timestamp} \\
        The time stamp of the event in the format `DD/MM/YYYY HH:ii:ss'
  \item {\bf reference} \\
        Whether or not the device is a reference device
  \item {\bf type} \\
        The type of call, one of `Call Drop', `Setup Failure', `Successful Call'
  \item {\bf start\_rat} \\
        The Start RAT of the event
  \item {\bf end\_rat} \\
        The End RAT of the event
  \item {\bf start\_mix\_band} \\
        The Start MixBand of the event
  \item {\bf start\_rrc\_state} \\
        The Start RRC State
\end{itemize}

~\\
A single KML file can contain any number of Placemark values, and must be 
wrapped inside the KML and Document tags, as show below. 

\begin{lstlisting}[language=XML]
<kml xmlns="http://www.opengis.net/kml/2.2">
  <Document>
    <name>KML Filename</name>
    <Placemark>
      ...
    </Placemark>
    <Placemark>
      ...
    </Placemark>
    <Placemark>
      ...
    </Placemark>
    <Placemark>
      ...
    </Placemark>
  </Document>
</kml>
\end{lstlisting}