\section{Product Evaluation}
\label{sec:product_evaluation}

The Cluster Analysis Tool that was developed as part of this final year project
could be regarded as a success. Although there are a number of improvements and
extensions that could be made, it is clear that when comparing the final 
product to the original user requirements set by BlackBerry (section 
\ref{sec:userrequirements}), there are little differences.

The Cluster Analysis Tool successfully parses a KML file and converts each of 
the given placemarks into Call Events. All call events can be clustered into 
various clusters by utilising an implementation of the DBSCAN algorithm.

Once the clustering process has been completed, the tool is able to provide a 
number of analysis options, and is ultimately output the analysis in various 
graphical formats.

As well a producing graphical outputs, the tool is also able to output to a 
KML file. Once opened in Google Earth, the KML file is able to display the 
various clusters, along with the radius around the centroid of each cluster. 

Each event is able to be clicked upon for more information, as well as allowing
the end user to ``overlay'' multiple KML files to view multiple weeks and/or 
multiple products.

Although a Graphical User Interface could have been developed, it was decided 
to keep the tool as a command line tool. This allows for a simple integration
with current internal BlackBerry systems, and it allows for an automated 
approach to utilising the tool (as well as a manual approach).

As it stands the product does meet all of the original requirements, however 
there are a number of improvements that could be made to further better the 
product.

Firstly the test route could shown within the output KML file. This would allow
for each event to be highlighted upon the route. In order for this to be 
achieved, the route would have to be added to the original KML file. 

Adding the route would not change the analysis or clustering processes, however
it would improve the overall link between the route and the call events.

Additionally, an overall area percentage figure could be introduced. This 
percentage would be the number of drops multiplied by the area coverage for a
given geographical coordinate. Typically the area coverage is 5m$^2$ (25m). 

This would mean that if there were 20 events within a 1KM radius, then the 
total area percentage representing events would be calculated as:

\begin{center}
  $ \large \frac{20 \times 0.25}{\pi \times 1^2} \times 100 = 15.9\% $
\end{center}

Effectively, this means that 15.9\% of the given cluster (in terms of area) 
represents call events.

~\\

Overall, the product meets the required specifications of this project. More 
specifically, the product has managed to ensure that the ``Must'' and 
``Should'' requirements are completed with some of the ``Could'' requirements
completed as well.
