\section{Personal Reflection}
\label{sec:personal_reflection}

The original proposal for this project was put forward by Anup (Call 
Performance Team Leader, BlackBerry) during the final month of my industrial 
placement. 

I was particularity interested in this project, as it was a follow on project 
from some of the work that I had completed upon placement.

Initially I was aware that I would have to produce some form of algorithm that
would cluster the calls in to groups. I was also aware that in some form or 
another I would have to define exactly what a cluster is, and how the algorithm 
could `detect' a cluster.

For me this posed an interesting problem, to which I would be able to solve 
though using standard computing practises. 

As a Software Engineer, I have to actively embrace the principles of 
engineering in order to analyse, design, develop and test software systems. 
This problem had to opportunity to test all of these key areas.

I found the research aspect of the problem incredibly interesting. Not only was
it technically based, it was also heavily philosophically based. The majority 
of the background research was obtained when I was upon placement, with further 
academic references added in over the due course of the project.

However I did found it interesting that the cluster analysis research was just 
as philosophical as it was technical. From a technical point of view there were
little arguments and deviations. However this was quite the opposite when some 
of the philosophical aspects of cluster analysis were discovered.

As an example, I found that the definition of a cluster was not only ambiguous,
it also had a philosophical nature. One individual could state that a dataset 
had a certain number of clusters, where as another individual could state that
figure to be something completely different.

These differences in opinions could have been obtained from different 
understandings of the original data set, to even the visual representation of 
the data set. 

I found out that it was simply the definition of a cluster that was the hardest
aspect of the field, not what I had originally expected which was the 
development and implementation of the algorithm.

I found the majority of the development straight forward, which was down to a 
number of major aspects. 

Firstly, BlackBerry activity engaged in the analysis and development stages of 
the project. This ultimately ensured that the end product was correct, and that
it met all of the system requirements.

Ideas were constantly bounced back between myself and Anup to try solve any 
minor problems that might have occurred during development. 

As previously mentioned, it also allowed for an industry quality circle to be 
formed, which ensured that the quality was there, and that the objectives were
always being met.

Secondly as I had an understanding of the mobile network, and some of the 
analysis activities that took place within BlackBerry, I was able to utilise 
my previous placement experience to leverage any work. 

An example of this can be found in which the way the analysis is shown. Call 
drop events are shown as red, and setup failure events are show as yellow. It 
is this common theme, that allows BlackBerry test engineers to instantly 
recognise what the analysis is trying to tell them.

Finally, as previously mentioned this project was a follow on project from my 
placement year. This fundamentally meant that I was able to start earlier. 

It was pushing the start time of the project forward, that enabled me to ensure 
that all research was conducted thoroughly, and that the requirements had been
correctly analysed and mapped to development targets.

It also provided some additional ``breathing room'' when other university 
commitments had to take priority.

~\\

I can certainly say that I have thoroughly enjoyed this project, partially the 
academic research side of the project. Consequently, my academic research 
skills have managed to improve substantially. 

I am also able to state that I have been able to re-enforce a second 
programming language in such a way that I could classify myself as an advanced 
user. 

As well as re-enforcing known development technologies, I have also been able
to learn new technologies. 

It is the combination of new skills learnt and the further improvement of 
existing skills that I will be able to cherish for future projects.