\section{Project Evaluation}
\label{sec:project_evaluation}

In total, forty-four weeks was formally set a side to complete this project.
However, I have managed to complete the project fully in thirty-nine weeks. It 
is clearly obvious that the time management skills used throughout this project 
helped to deliver the project well within the allocated time.

At every major stage (such as research, analysis, implementation etc) a high 
quality piece of work has been initially produced, refined and improved through
client, supervisor and examiner feedback.

The largest amount of time spent was compiling the research section. Overall 
three major areas of computing --- mobile networks, clustering and 
visualisation techniques --- were managed to be researched in a great amount 
of depth. It is this additional understanding that was able to be applied in 
some of the key sections found later on, such as the design and implementation 
stages.

Development time was considerably lower than what I had originally planned for.
There are several reason for this. Firstly due to the increased time that I had
managed to obtain through starting the project early, I was able to spend more 
time upon the Research and Design sections. This allowed for a more, rounded
well thought out approach to the project .

Secondly the design section of the project was carried out with careful 
consideration. This was to ensure that all aspects were covered, to ensure that
the best possible design and implementation could have been produced.

Finally, the development strategy was to use an incremental evolutionary 
prototype approach. This is a hybrid of the incremental strategy and the 
evolutionary strategy. The reason for this is that it allowed for each MoSCoW 
objective to be developed independently, and to be able to receive direct 
feedback from BlackBerry.

The modules that were developed help to keep the overall testing time down. The
reason for this is that each module could be tested independently, and did not
require any dependencies from other modules. Furthermore, testing was conducted
as development occurred. This highlighted a large percentage of bugs, and were 
fixed immediately.

To ensure the project was on track various meetings took place throughout the 
year. 

A weekly project meeting in the style of a `quality circle' was attended in by 
myself, my supervisor and other group peers. These meetings provided an 
opportunity for various topics of discussion to occur. The topics included new 
ideas that could be integrated into the product or project, as well as 
providing feedback upon current ideas. 

A bi-weekly product meeting was conduced via a conference call between myself 
and Anup who is the Call Performance Team Leader for BlackBerry.

Anup would provide feedback upon any development work that was conducted, as 
well as discussing any future functionality. These calls proved to be useful, 
as it allowed for the bridge between an academic project and an industry 
product to take place.

As part of the integration strategy, I was asked to conduct a presentation upon
the project as whole to BlackBerry. This allowed for various research topics to
be presented, as well as the design principles behind the software product to 
be highlighted. 

As part of the mid-term review of the project, a poster was used to consolidate
all the research and design ideas obtained up to that point. The poster was 
presented to the various members of academic staff, research staff as well as 
other peers. This allowed for valuable feedback to be obtained from a wide 
audience, as well discussing current progress and future expected progress.

The aims of this project were to demonstrate that is is possible to cluster 
mobile call coverage figures, as well as analysing these figures to provide an 
in depth understanding of call performance. After completing this project and 
gaining valuable feedback from BlackBerry, it is clearly evident that the 
project can be classified as a success.