\subsection{Density-Based Methods}
Density-based algorithms try to locate regions that are of high density and 
separate these regions from the data space \citep{tan05,han06}. Regions that 
are of a low density are either ignored, or removed completely from the data 
space \citep{han06}.

Unlike the previous algorithms discussed, clusters can be of any arbitrary 
size, as a cluster is defined as dense regions of objects (rather than 
similarity between two objects).

\subsubsection{Density-Based Spatial Clustering of Applications with Noise}
\label{sec:DBSCAN}
The DBSCAN algorithm is a centre-based, density-based algorithm. The algorithm
is able to identify regions of high density into clusters, of any arbitrary 
shape \citep{han06}.

Density is estimated for a given object by counting the number of objects 
within the neighbourhood of a radius, {\em Eps} \citep{tan05}. A cluster is 
defined as a maximal set of density-connected objects. 

In order to fully define the DBSCAN algorithm, additional terms will be 
required to be defined. These terms are specific to the DBSCAN algorithm, and 
are listed below.

\begin{itemize}
  \item A core object is an object that falls within the {\em Eps} of a given 
        radius and when grouped with other points, the total number of points 
        exceeds the minimum points \citep{han06}.
  \item A border object is an object that falls within the neighbourhood of a 
        core object. A border object can fall in the neighbourhoods of more than 
        more core object. A border object is not a core object \citep{han06}.
  \item A noise object is any object that is not classified as either a core 
        object or as a border object \citep{han06}.
  \item The neighbourhood within a radius, {\em r}, of a given object is called 
        the r-neighbourhood of the object \citep{tan05}.
  \item An object is referred to as a core object if the r-neighbourhood of that 
        object contains at least a minimum number of objects \citep{tan05}.
  \item Object x is directly density-reachable from object y if object x is 
        within the r-neighbourhood of object y and object y is a core object
        \citep{tan05}.
\end{itemize}

The DBSCAN algorithm classifies an object as being:
\begin{enumerate}
  \item within a dense region (core object); {\bf \em or}
  \item on the edge of a dense region (border object); {\bf \em or}
  \item in a sparsely occupied region (noise object).
\end{enumerate}

Informally, the DBSCAN algorithm could be described as any two core objects 
that are within a distance, {\emph Eps}, of each other are placed into the same 
cluster. A border object that is close enough to a core object is placed into 
that cluster, with all noise objects being discarded. 

\paragraph*{DBSCAN Formal Algorithm}
\subparagraph*{Input:}
\begin{itemize}
  \item {\bf D} - the data set of {\em n} objects
\end{itemize}

\subparagraph*{Output:}
\begin{itemize}
  \item A set of clusters
\end{itemize}

\subparagraph*{Method:}
\begin{enumerate}
  \item Label all objects as core, border or noise objects within D;
  \item Remove the noise objects from D;
  \item Calculate the distance between all core objects that are within 
        {\em Eps} of each other;
  \item Make each group of connected core objects into their own cluster;
  \item Assign each border object to the closest core object’s cluster.
\end{enumerate}

One of the fundamentals of the DBSCAN algorithm is the notion of density 
reachability. It is because of this that the algorithm is able to automatically
detect clusters of objects, rather than having to specify the number of 
clusters --- unlike the partitioning methods --- \citep{han06}. 

If an object is unable to be placed into a cluster, it is classed as noise, thus 
improving the overall quality of the clusters.

The single-link effect (which is common in partitioning methods) is reduced 
dramatically \citep{tan05}, due to the inclusion of the minimum points value; 
this ultimately allows the algorithm to find clusters of any arbitrary shape
\citep{han06}. 

However the minimum points value is fixed at the start of the clustering 
process, and can cause problems with datasets that have large differences in 
densities \citep{han06}.

The algorithm is (generally) insensitive to the ordering of the objects in the
dataset. However if an object is sitting on the edge of two clusters, the 
membership to either cluster may change \citep{han06,tan05}.

Finally, the quality of the clusters depends upon the distance measuring 
method. The Euclidean distance calculation is normally used. This is based upon
Pythagoras' theorem, and utilises an ``as-the-crow-flies'' approach to finding
the distance between two points.