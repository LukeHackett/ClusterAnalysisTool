\subsection{Partitioning Methods}
A partitioning algorithm will organise {\emph n} number objects within a 
dataset, into {\emph k} non-overlapping partitions (clusters) as long as 
{\emph k} $\leq$ {\emph n} \citep{han06}. 

This means that each object will belong to one cluster and one cluster only. 

The objects in a cluster are more similar with each other, in comparison to 
objects that belong to other clusters. These are regarded as being dissimilar 
in terms of the data set attributes \citep{tan05}.

\subsubsection{K-Means}
The K-Means algorithm will partition a set of objects into {\emph k} number of 
clusters in such a way that the similarity of the objects that make up a 
cluster is high, but the similarity of two or more separate clusters (and their 
objects) is low.

Cluster similarity is measured with regards to the average (mean) value of all 
objects in a given cluster. This average value is known as the cluster's 
centroid.

The K-Means algorithm will firstly randomly select {\em k} number of objects 
from the dataset and put them into their own cluster \citep{han06}. For each of
the remaining objects in the dataset, an object is assigned to the cluster that
it is most similar to \citep{tan05}.

The similarity is based upon the distance between the object in question and 
the centroid object of a given cluster. Once the pass is completed, the 
centroid of all clusters is recomputed to gain the new mean cluster 
\citep{tan05}. 

This process will continue to run until there have been no object moves 
detected \citep{han06}.

\paragraph*{K-Means Formal Algorithm}
\subparagraph*{Input:}
\begin{itemize}
  \item {\bf K} - the number of clusters
  \item {\bf D} - the data set of {\em n} objects
\end{itemize}

\subparagraph*{Output:}
\begin{itemize}
  \item A set of k clusters {\em K}
\end{itemize}

\subparagraph*{Method:}
\begin{enumerate}
  \item randomly choose k objects from D;
  \item {\bf repeat}
  \item \begin{list}{$\square$}{\leftmargin=1em \itemindent=0em}
          assign each object to the cluster to which it is most similar to the 
          cluster’s centroid;
        \end{list}
  \item \begin{list}{$\square$}{\leftmargin=1em \itemindent=0em}
          update the cluster centroid for each cluster;
        \end{list}
  \item {\bf until} no change;
\end{enumerate}

Computationally speaking, the K-Means algorithm maybe faster than hierarchical 
clustering methods, if the number of cluster, {\emph k}, is small \citep*{ios}.
The reason for this is that the algorithm utilises the centroid as a 
comparative object. 

The algorithm can also produce more tightly packed clusters in comparison to 
hierarchical based clustering \citep{ios}.

However comparing the accuracy of the clusters that have been produced can be 
difficult, especially as the number of clusters, k, affects the outcome. In order 
to improve the accuracy of the results, it may be required to run the algorithm 
several times with different k value (\citep{ios}).

\subsubsection{K-Medoids}
One of the main issues faced with the K-Means algorithm, is that it is highly 
sensitive to outliers, because the centroid object is the mean of all the 
objects within that cluster \citep{tan05}. It is possible to modify the K-Means
algorithms to remove such sensitivity \citep{tan05}. 

Instead of creating a mean centroid object based upon all the objects within 
that cluster, we could pick an actual object to represent the centroid. Each 
remaining object would then be clustered with the centroid object that is it 
most similar to \citep{tan05}.

The algorithm would iterate until each centroid is actually the medoid of its 
cluster. It is this idea that forms the basis of the K-Medoids algorithm for 
grouping a set of objects into {\em k} clusters \citep{everitt74}.

The initial centroid objects are chosen at random According to \citep{tan05}. 
The process of replacing the current centroid with another object continues as 
long as the resulting clustering is improved \citep{tan05}. 

The quality of the clustering is measured using a cost method, which will 
measure the average dissimilarity between an object and the centroid of the 
cluster \citep{tan05}.

\paragraph*{K-Medoids Formal Algorithm}
\subparagraph*{Input:}
\begin{itemize}
  \item {\bf K} - the number of clusters
  \item {\bf D} - the data set of {\em n} objects
\end{itemize}

\subparagraph*{Output:}
\begin{itemize}
  \item A set of k clusters {\em K}
\end{itemize}

\subparagraph*{Method:}
\begin{enumerate}
  \item 1.  randomly choose {\em k} objects from D as the initial centroids;
  \item {\bf repeat}
  \item \begin{list}{$\square$}{\leftmargin=1em \itemindent=0em}
          assign each remaining object to the cluster to the nearest cluster, 
          based upon the centroids;
        \end{list}
  \item \begin{list}{$\square$}{\leftmargin=1em \itemindent=0em}
          randomly choose a non-centroid object, {\em o};
        \end{list}
  \item \begin{list}{$\square$}{\leftmargin=1em \itemindent=0em}
          compute the total cost of swapping the centroid object with, {\em o};
        \end{list}
  \item \begin{list}{$\square$}{\leftmargin=1em \itemindent=0em}
          if total cost $<$ 0 then swap the centroid with {\em o};
        \end{list}
  \item {\bf until} no change;
\end{enumerate}


The K-Medoids proves to be a robust algorithm in the presence of noise and/or 
outliers. The reason for this is because the medoid is not influenced by extreme 
values, as found with the mean value.

The algorithm is efficient for small sets of data, but can be inefficient in 
terms of time when working with large data sets.
