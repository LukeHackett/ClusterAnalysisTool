\subsection{Graph Design}

There are four `rules' to apply when choosing a graph format \citep{kosslyn06}.
If the wrong type of graph is chosen then the graph will not communicate with 
the reader effectively --- no matter how pretty it looks. 

The four `rules' are:
\begin{enumerate}
  \item Use a graph to illustrate relative amounts;
  \item Specify the subject;
  \item Present the data needed for a specific purpose;
  \item Use known concepts and formats.
\end{enumerate}


\subsubsection*{Illustrate}
A graph should only be used to illustrate relations amongst data 
\citep{kosslyn06} because graphs use variations within a visual dimension. This
allows us to recognise that one bar in a given bar chart is larger than another
bar in the same chart. 

\begin{quote}
  ``our perceptual systems allow us to quickly to detect difference among 
  heights, or the slope of lines. However, they do not allow us to register 
  absolute heights or slopes very well.'' \citep{kosslyn06}
\end{quote}

This is the fundamental reason, as to why graphs should illustrate the data, 
rather than allowing the reader to automatically obtain knowledge or wisdom.

In order to recognise specific values, the reader has to constantly refer back 
to the graph's scale, and make comparisons between each bar or slope. This 
will not only become annoying, but it will remove the focus from the data and 
towards the lesser important scale.


\subsubsection*{Specify}
In order for a reader to understand what the display is informing them of, they
need to know what it is the display is trying to inform them of 
\citep{kosslyn06}. This can be achieved by utilising a simple, precise, 
descriptive title.

Title of a display must be thought of before creating the display. This will 
help to decide what is useful and relevant within the display \citep{kosslyn06}.

An example could be ``Plant Productivity, 1995 -- 2000'', which would lead you 
to include data for each of the years, whereas a title such as ``Number of 
Units Produced in the United Kingdom'' would lead you to include data that is 
the average annual production figures for the United Kingdom.


\subsubsection*{Purpose}
A graph should allow people to answer specific questions. The nature of the 
question, and the information to answer that question needs to be specific 
\citep{kosslyn06}.

``People expect a question to be answered with the appropriate amount of 
information. No more and no less.'' \citep{grice75}. Readers use displays to 
answer questions, and require that there questions are answered in the context 
of the display. Hence why the display must have a purpose.


\subsubsection*{Formats}
``Displays are designed to communicate to a particular audience'' 
\citep{kosslyn06}. The concepts that are presented to the audience must be 
familiar and recognisable to the audience. Terminology, language and notations
must be followed within the display otherwise the audience may become confused.

Readers can only interpret a display if they have already stored the necessary
background in memory \citep{kosslyn06}.

Once the audience and the numbers that are to be displayed have been thought 
of, then the consideration of particular graphing formats can begin 
\citep{kosslyn06}. 

The chosen graph will depend upon the type of data that is being displayed. 
Within this section, a number of ``classic'' graphing displays will be 
discussed.
