\chapter{Introduction}

There are over 80 million mobile subscriptions in the UK, with ninety-two 
percent of UK adults owning a mobile phone \citep{ofcom:online}. It has been
estimated that the total throughput of the UK mobile network is between thirty
and forty Gigabits per second \citep{webb10}. 

The mobile network is something that is actively used in every day lives. To 
ensure that it is always available to it's users, the mobile network has to be 
kept as efficient as possible.

Consider a mobile device {\em consistently} failing to connect to the mobile 
network at a given geographical location. The question that many test engineers 
may want to be able to answer is:

\begin{quote}
  ``Does this event occur upon a regular basis or is it just an `insignificant 
    phenomenon'?'' 
\end{quote}
~\\
Now consider another mobile device, in a different geographical location, that 
is able to connect to the mobile network, but is unable to maintain it's 
connection to the network. Test engineers may want to find out if:

\begin{quote}
  ``This a network issue, is this a hardware issue, or a software issue?''
\end{quote}
~\\
These are just two of many questions that are actively asked by test engineers, 
and when a device occurs problems like these it is up to the manufacturer of 
the device to investigate the problems, and fix them.

This project will simply try to answer the question
\begin{quote}
  ``Is it possible to cluster and analyse call logs generated from the global 
    drive test process?''
\end{quote}
~\\
The overall aim of this project is to try to cluster the call data into 
comparable clusters, that allow for easy analysis techniques to take place. The 
project will show that data can be clustered from many aspects, and will also 
highlight that the visualisation of data is just as powerful as the initial 
clustering process is.

The research section of this report will describe the various methods used to 
cluster data, and how the data can be correctly and accurately reported back to 
the end user. Based upon this research and a set of user requirements directly 
from BlackBerry a Product Specification was able to be implemented.

The design and implementation section of the report will highlight the various 
`journeys' made when initially designing and analysing the given problem. The 
implementation section will focus upon various important aspects of the project
allowing for a more in-depth detailed explanation of the system, from a 
technical perspective.

The testing and maintenance sections will describe the test strategy that was 
used, along with the various tests that were run. The maintenance section 
focuses upon any future enhancements or work in progress items that could be 
implemented.

Finally, both the project and the final product will be evaluated.
