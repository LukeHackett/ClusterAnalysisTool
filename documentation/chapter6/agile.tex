\section{Agile}

The agile software development methodology is designed to "reduce risk by 
delivering software systems in short bursts or releases" \citep{dawson09}. 

The concept behind the agile software development is to reduce the total weight
time that clients have to endure, by releasing working systems in as little 
time as possible. 

Each iteration may not contain a full working system, but it will contain a 
partial system, that could be used by the client. Agile methods are suited to 
projects that have unclear or rapidly changing requirements \citep{dawson09}.

\subsection{Advantages}
\citet{dawson09} states that the main principles of the agile software 
development methodology is as follows:

\begin{itemize}
	\item The methodology surrounds the concept of regular face-to-face meetings
        (as opposed to in-depth documentation);
	\item A close working relationship between the client and the developers;
	\item A short iterative time scale (usually weeks rather than months or 
        years);
	\item Easily able to change the requirements at any stage.
\end{itemize}

\subsection{Disadvantages}
However, \citet{dawson09} also states that the agile software development 
methodology does have it's problems:
\begin{itemize}
	\item There can be a limited amount of documentation, as this step is usually 
        skipped to save time;
  \item The uncertainty of a specification may lead to poor code and/or 
        structure.
\end{itemize}