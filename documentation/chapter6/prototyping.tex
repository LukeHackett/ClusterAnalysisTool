\section{Prototyping}
Prototyping is used when there is uncertainty about the technical solution to 
a problem \citep{dawson09}. 

It can be often useful to perform a number of experimental prototypes in order 
to gain understanding and knowledge, such as working on a new hardware platform
or developing a new algorithm. 

The prototyping software development methodology allows for many ideas and 
theories to be implemented, tested and evaluated. There are two main forms of 
prototyping, throw-away prototyping and evolutionary prototyping.

\subsection{Throw-away prototyping}
Once a prototype has been developed, the product associated with the prototype 
could be thrown away, with the actual development of the desired product 
starting from a blank ``canvas''.

\subsubsection{Advantages}
\citet{knott_dawson99} mention the following list of advantages over other 
software development methodologies: 
\begin{itemize}
  \item Errors and omissions in the requirements specification can be quickly 
        fixed;
  \item An artificial is produced quickly (keeping the client happy);
  \item The prototype is able to test the feasibility of the product;
  \item Alternatives of a various prototypes can be compared;
  \item Allows for improvement of communication between the developer and the 
        client.
\end{itemize}

\subsubsection{Disadvantages}
\citet{dawson09} highlights that there are a number of issues with the 
throw-away prototyping methodology, as highlighted below:
\begin{itemize}
  \item The prototype may contain a number of bugs;
  \item The prototype might look good, but technically could be poor. If it was 
        integrated into the system, then this could lead to poorly structured 
        code.
  \item Prototype development within a different language or system to the 
        final implementation may allow for technical differences to occur.
\end{itemize}

\subsection{Evolutionary prototyping}
Unlike the throw-away prototyping, a prototype could be developed further into
the final desired product. Utilising this idea, is known as evolutionary 
prototyping.

\subsubsection{Advantages}
The advantages are generally the same as the advantages found within the 
throw-away prototyping methodology with the following addition:
\begin{itemize}
  \item Any code that is developed, is reused and improved.
\end{itemize}

\subsubsection{Disadvantages}
As with the advantages, the disadvantages are generally the same as the 
advantages found within the throw-away prototyping methodology with the 
following addition:
\begin{itemize}
  \item A well structured design of code is required so that it can be 
        developed and allow for the evolution process to occur.
\end{itemize}