\section{Summary}
In order to achieve the best product possible, it is clearly evident that the 
client must be kept in the loophole as often as possible. This allows for any 
revisions, modifications, and changes to be considered and implemented with as 
little delay as possible.

Although this project has some well defined requirements, the details of the 
requirements are incomplete in some areas. However some key details have 
already been set, such as the programming language to be used and the target 
system. 

The rule of thumb states that if uncertainty is high then the use of an 
evolutionary approach is recommended \citep{hughes_cotterell09}. As well as 
this, the incremental approach seems to be an ideal methodology to utilise. 
The incremental methodology is ideal for projects that have a tight schedule 
\citet{hughes_cotterell09}.

Although the schedule of the project is not necessarily tight, it does contain 
a lot of potentially time consuming design research. By utilising the 
incremental methodology as a fundamental methodology, this should allow for 
small incremental developments to occur, with little work to re-complete if the
original path is later deemed to be incorrect.

Within each increment, an evolutionary prototype approach will be taken. The 
main reason for this is that it allows for development and research to coexist 
side by side. If the research pays off, then it's output can directly be 
utilised within the final published product.